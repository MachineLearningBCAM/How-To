\documentclass[a4paper]{article}
\usepackage[utf8]{inputenc}
\usepackage[spanish]{babel}
\usepackage{graphics}
\usepackage{subfigure}
\usepackage{hyperref}
\usepackage[backend=biber,style=apa]{biblatex}
\usepackage[usenames]{color}
\usepackage{xcolor}
\usepackage{listings}
\usepackage{comment}

\lstset{columns=fullflexible, basicstyle=\ttfamily,
	backgroundcolor=\color{light-gray},xleftmargin=0.5cm,frame=tlbr,framesep=4pt,framerule=0pt}
\lstset {
	language=python,                  % language code
	basicstyle=\footnotesize,      % font size
	numbers=none,                  % where to put line numbers
	numberstyle=\footnotesize,     % numbers size
	numbersep=5pt,                 % how far the line numbers are from the code
	backgroundcolor=\color{white}, % background color
	showspaces=false,                          % show spaces (with underscores)
	showstringspaces=false,            % underline spaces within strings
	showtabs=false,                            % show tabs using underscores
	frame=single,                  % adds a frame around the code
	tabsize=4,                     % default tabsize
	breaklines=true,                  % automatic line breaking
	columns=fullflexible,
	breakautoindent=false,
	framerule=1pt,
	xleftmargin=0pt,
	xrightmargin=0pt,
	breakindent=0pt,
	resetmargins=true
}
\title{Project structure - Python}
\date{October  2020}
\author{BCAM}

\begin{document}
	\maketitle
	\tableofcontents
	\section{Directory tree}
		The directory tree and its files should look like the following:\\
		\begin{itemize}
			\item example\_pkg
			\begin{itemize}
				\item LICENSE.txt
				\item README.md (or README.txt)
				\item requeriments.txt
				\item tests
				\begin{itemize}
					\item test\_basic.py
					\item test\_special.py
				\end{itemize}
				\item docs
				\begin{itemize}
					\item images
					\begin{itemize}
						\item img1.png
						\item img2.png
					\end{itemize}
					\item paper.pdf
					\item usage.*
					\item faq.*
				\end{itemize}
				\item example\_pkg
				\begin{itemize}
					\item main.py
					\item scripts
					\begin{itemize}
						\item load\_data.py
						\item graphics.py
					\end{itemize} 
				\end{itemize}
				
			\end{itemize}
		\end{itemize}
	\section{Main packpage}
		\subsection{README}
			We should have a readme file that contains information that is commonly required to understand what the project is about, the file is usually  "README.txt" or "README.md"\\
			The file should use UTF-8 encoding, typical contents for this file would include an overview of the project, basic usage examples, etc. Generally, including the project changelog in here is not a good idea, although a simple “What's New” section for the most recent version may be appropriate.\\
			Keep in mind that a README file can be too long as we want, and too long is better than too short.\\
			We can take a look at some extra tips on \url{https://github.com/paperswithcode/releasing-research-code}, so here we recommend that at least we can find in README file:
			\begin{itemize}
				\item Name
				\item Badges (optional)\\
					About License, Software, Social network, Downloads...
				\item Website (optional)\\
					Can be a link to the paper
				\item Version (optional)\\
				\item Description
				\item Visuals\\
					It could be helpful add screenshots or videos to complete the description and what our code does
				\item Installation / Requirements\\
					How the user can install the library and whether there are dependencies
				\item Training\\
					How to use the library for training, the necessary parameters and/or inputs, explained in detail to help the end user use it\\
				\item Evaluation\\
					Explain how the model is evaluated, that is, in a classifier would be the "predict" method, and the measures you want to add inside the evaluation section.\\
				\item Pre-trained models\\
					In case of publishing algorithms where we can leave an already trained model, we will indicate it in this section.
					The aim is to test our model in the easiest possible way, therefore, the user will only load the model and run the "predict" method.\\
				\item Results\\
					Here we will indicate the results we have achieved, we will be able to make a comparison, so that our success in the published project is easily recognizable.\\
			\begin{comment}
				\item Examples (optional)\\
					 Only in case we think it is necessary, we can add this section, but mainly with the previous ones it is enough.
					 Also if we have the examples in the folder "tests" the user will go directly there to see the examples.
			\end{comment}
				\item Support \\
					where they can go to for help, email address...
				\item Authors
			\begin{comment}
				\item Road map (optional):\\
					If we've ideas for releases in the future
				\item Contributing (optional)\\
					State if you are open to contributions and what your requirements are for accepting them.
			\end{comment}
				\item License\\
					Choose a license and indicate it here, it's a good idea add the badge as well, but it's good to make it clear in a separate section
				\item Citing\\
					Leave as simple as possible the steps to quote both our code and if we have a paper
			\begin{comment}
				\item Project status\\
					We can write a section or put a badge or note at the top of the README
			
				\item References (optional)\\
			\end{comment}
			\end{itemize}
			\subsubsection{Badges}
				It is very visual way to indicate for example the license, the version of our software, the version of the programming language required (or merely the programming language), the number total of downloads/installations per week/month (or last week/month), link to a social network account (such as Twitter, Reddit or YouTube), indicate whether we continue working on the repository or it's finished, whether we continue to provide a maintenance to the software we have published, the quality of the code, whether it has any vulnerability detected, whether the publication is open source...\\
				In short, you can create the badge you want to give fast and accurate information to a potential user, the website \url{https://shields.io/}  can help.\\
				We can see a repository of badges: \url{https://github.com/Naereen/badges/blob/master/README.md}\\
		\subsection{LICENSE.txt}
			This tells users who install our package the terms under which they can use our package.\\ The one that is more permissive corresponds to the MIT license, on the other hand, if we consider that the uploaded software has a potential use, it would be convenient to use the GNU GPLv3 license (it can change with new versions in the future)\\
			So, to create the file "license.txt" we've just copy and paste the template updating some details like the year, fullname... and leave the file in the root directory.\\
			We can add easily the file license.txt with GitHub following this link: 
			\url{https://docs.github.com/en/github/building-a-strong-community/adding-a-license-to-a-repository}\\
			A website that helps to choose the license, having templates for each of them is \url{https://choosealicense.com} \\
			\subsubsection{MIT License}
				Permissions with this license:
				\begin{itemize}
					\item Commercial use: The licensed material and derivatives may be used for commercial purposes
					\item Distribution: The licensed material may be distributed
					\item Modification: The licensed material may be modified
					\item Private use: The licensed material may be used and modified
				\end{itemize}
				Conditions with this license:
				\begin{itemize}
					\item License and copyright notice: A copy of the license and copyright notice must be included with the licensed material
					content...
				\end{itemize}
				Limitations with this license:
				\begin{itemize}
					\item Liability: This license includes a limitation of liability
					\item Warranty: This license explicitly states that it does NOT provide any warranty
					content...
				\end{itemize}
				
				Here is the MIT license template, so we've just to change the [year] and [fullname]:
				
				\begin{lstlisting}[caption=MIT license template, label=lst:mitLicenseTemplate]
				MIT License
				
				Copyright (c) [year] [fullname]
				
				Permission is hereby granted, free of charge, to any person obtaining a copy
				of this software and associated documentation files (the "Software"), to deal
				in the Software without restriction, including without limitation the rights
				to use, copy, modify, merge, publish, distribute, sublicense, and/or sell
				copies of the Software, and to permit persons to whom the Software is
				furnished to do so, subject to the following conditions:
				
				The above copyright notice and this permission notice shall be included in all
				copies or substantial portions of the Software.
				
				THE SOFTWARE IS PROVIDED "AS IS", WITHOUT WARRANTY OF ANY KIND, EXPRESS OR
				IMPLIED, INCLUDING BUT NOT LIMITED TO THE WARRANTIES OF MERCHANTABILITY,
				FITNESS FOR A PARTICULAR PURPOSE AND NONINFRINGEMENT. IN NO EVENT SHALL THE
				AUTHORS OR COPYRIGHT HOLDERS BE LIABLE FOR ANY CLAIM, DAMAGES OR OTHER
				LIABILITY, WHETHER IN AN ACTION OF CONTRACT, TORT OR OTHERWISE, ARISING FROM,
				OUT OF OR IN CONNECTION WITH THE SOFTWARE OR THE USE OR OTHER DEALINGS IN THE
				SOFTWARE.
				\end{lstlisting}
			\subsubsection{GNU GPLv3}
				The GPL is based on four freedoms: the freedom to use the source code for any purpose, the freedom to make modifications, the freedom to share the source code with anyone, and the freedom to share changes.\\
				GPL does not prohibit users from selling derivative works that are based on the original source code, it merely requires source code to be freely available to anyone who wants it. This is the "reciprocity obligation."\\
				If our code is published under license X, any user will be obliged to publish his modifications under the same license (it is not possible to publish under license MIT for example)\\
				Permissions with this license:
				\begin{itemize}
					\item Commercial use: The licensed material and derivatives may be used for commercial purposes
					\item Distribution: The licensed material may be distributed
					\item Modification: The licensed material may be modified
					\item Patent use: This license provides an express grant of patent rights from contributors
					\item Private use: The licensed material may be used and modified
				\end{itemize}
				Conditions with this license:
				\begin{itemize}
					\item Disclose source: Source code must be made available when the licensed material is distributed
					\item License and copyright notice: A copy of the license and copyright notice must be included with the licensed material
					\item Same license: Modifications must be released under the same license when distributing the licensed material. In some cases a similar or related license may be used
					\item State changes: Changes made to the licensed material must be documented
				\end{itemize}
				Limitations with this license:
				\begin{itemize}
					\item Liability: This license includes a limitation of liability
					\item Warranty: This license explicitly states that it does NOT provide any warranty
				\end{itemize}
		\subsection{requirements.txt}
			It should specify the dependencies required to contribute to the project: testing, building, and generating documentation.\\
			We'll install with this command:
			\begin{lstlisting}[caption=requirements.txt, label=lst:exampleRequeriments]
				pip install -r requirements.txt
			\end{lstlisting}
			We can simply leave the name of the library and the latest one will be downloaded, but if there are compatibility problems, we can specify the version with "==". Here is an example:\\
			\begin{lstlisting}[caption=requirements.txt, label=lst:exampleRequeriments]
				matplotlib==3.3.0
				numpy
				scipy==1.5.2
			\end{lstlisting} 
	\section{tests}
		We will store here all the available unit tests of our project.\\
		We do not want to test the code, but the whole library/program, making a practical use that will serve as an example for future users.\\
		We can create subfolders for each example as well.
	\section{docs}
		We can add additional documentation such as the user guide (user's manual, called "usage"), frequently asked questions (FAQ file, called "faq")\\
		Sometimes, we will be interested in creating folders to improve the organization, such as creating an image folder called "images".\\ 
		\subsection{images}
			Here we will put the image files, if we need a structure for better organization, for example to separate by formats or by type of content, we can create subfolders
	\section{Packpage}
		This folder will have the same name as the root, and will be the name of our software. Inside we will have our software structured in an organized way.\\
		Sometimes, our software will be composed of a main file and other secondary files that contain useful functions, in this case, it is advisable to create a folder called "scripts", to leave in this folder the main file of our software
		\subsection*{scripts}
			Here will be the files containing the functions that are called from the main file.\\
			These are usually functions grouped according to their functionality, such as "data loading", "graphics", "X value calculation", etc.\\
			If we have several functions that we do not know how to group them, a useful resource is to call the file "utils".\\
	\section{Example}
		Here we can see examples:\\
		\begin{itemize}
			\item \url{https://github.com/MachineLearningBCAM} \\
		\end{itemize} 				
		
\end{document}